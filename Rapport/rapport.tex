% Document description
\documentclass[a4paper,11pt]{report}
\usepackage[utf8]{inputenc}
\usepackage[T1]{fontenc}
\usepackage{lmodern}
\usepackage[francais]{babel}
\usepackage{listings}
\usepackage{graphicx} %Pour inclure les images
\usepackage{float} %Pour plus de précision sur le placement
\usepackage{color}
\usepackage[hidelinks]{hyperref} %Pour les liens dans le PDF
\usepackage{fancyhdr} %En-tête + pieds de page
\usepackage{lastpage}

% Metadata
\title{HMIN317 - Moteur de jeux \\ Compte-rendu TP2}
\author{BOYER Benoît}
\date{Septembre 2017}

\input{css.tex}


% --------------> Document beginning <--------------
\begin{document}

  	  %Pattern de Peter Wilson
  	  \begin{titlepage} % Suppresses displaying the page number on the title page and the subsequent page counts as page 1
	
	  \raggedleft % Right align the title page
	
	  \rule{1pt}{\textheight} % Vertical line
	  \hspace{0.05\textwidth} % Whitespace between the vertical line and title page text
	  \parbox[b]{0.75\textwidth}{ % Paragraph box for holding the title page text, adjust the width to move the title page left or right on the page
		
		  {\Huge\bfseries Compte-rendu TP2 \\[0.5\baselineskip] Game loops et timers}\\[2\baselineskip] % Title
		  {\large\textit{HMIN317 - Moteur de jeux}}\\[4\baselineskip] % Subtitle or further description
		  {\Large\textsc{BOYER Benoît}} % Author name, lower case for consistent small caps
		
		  \vspace{0.5\textheight} % Whitespace between the title block and the publisher
		
		  {\noindent M2 IMAGINA - Septembre 2017}\\[\baselineskip] % Publisher and logo
	  }

  \end{titlepage}
  
    \tableofcontents
	\pagebreak


    \section{Question 1}
    \subsection{Modifier votre TP précédent pour lire une height map}
    La classe MainWidget sert à gérer l'affichage et les évènements du programme, en effet si on se fie aux foncions présentes dans le .h :
    \lstinputlisting[language=C++, caption=Fonctions de la classe MainWidget, firstline=75, lastline=85]{../mainwidget.h}
    	
\end{document}